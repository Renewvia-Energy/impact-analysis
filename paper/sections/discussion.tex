\section{Discussion and Qualitative Insights}
\label{sec:discussion}
This study conducts a thorough analysis of solar mini-grids' impact in surveyed communities, examining changes across a wide range of variables in the first year post-installation. Utilizing both descriptive and inferential statistics, it highlights key observed changes. Although not exhaustive, the study captures diverse socio-economic benefits of electrification, shedding light on its potential positive impact on African rural communities. The findings are organized into themes: gender equality, productivity, health, safety, and economic activity, enabling detailed exploration of implications and study limitations. This approach aims to comprehensively understand the multifaceted impact of solar mini-grid implementation.

In this context, gender equality was defined as increased equal access to education for both boys and girls, enhanced economic opportunities for women, and reduced time spent on household chores typically assigned to young and adult females in those communities. There were 11 questions pertaining to this KPI from which several variables were extracted for analysis. The results indicate a notable and significant improvement ranging from access to education for boys and girls alike to increased economic inclusion for women. The associated investment into the installation and capacity of the mini-grid proved to be positively correlated with higher school enrollment rates across communities.
Responses from open-ended survey questions added depth to these findings. Post-connectivity, school-aged children, especially girls, reported higher grades, attributed to extended study hours into the night and the ability to complete schoolwork more effectively. Additionally, the mini-grid connection facilitated access to the internet, providing children with additional learning materials. This connectivity proved particularly beneficial for girls, who gained increased exposure to and knowledge about feminine care and hygiene issues.
Furthermore, the reduction in time spent on chores like water and cooking fuel collection offered relief to the primary caregivers in the household, who were often young or adult women. Furthermore, the availability of nighttime lighting allowed for the completion of chores later in the evening, resulting in earlier meal preparations. The broader economic landscape also saw a transformation, with the mini-grid connection opening up new job opportunities from which women substantially benefited.

Before the mini-grid connection, many economically productive activities were often determined by the natural light hours and were limited by little-to-no access to electricity. For instance, many residents had to travel to charge electronic appliances such as phones and, so, restrict the time they could spend on other activities.
Respondents involved in the fishing industry noted that their fishing activities were limited due to the absence of ice for preserving their catch before reaching shore. Consequently, they restricted their daily catch to quantities they could sell immediately, as smoking was the only available method for preservation.
These inefficiencies were curtailed or eliminated by a reliable power source. More broadly, productivity is related to light hours for households, time spent on and distance traveled for specific activities like water collection and appliance charging, and enhanced operational hours and workforce capacity for organizations. The surveys contained 31 productivity-related questions for households and 7 for businesses. Analyses of the metrics related to productivity demonstrated gains and savings in terms of resources allocated to various activities, thus pointing to greater efficiency at the individual and community levels. Connecting to the mini-grid gave more light hours, reduced reliance on unclean or non-renewable power sources, and provided on-demand charging stations at home for all their electronic devices. \textcolor{purple}{This result affirms the conclusions of \cite{wassie2021socio, uwineza2021analysis}, where the installations of solar panels and a small hydroelectric power plant in Ethiopian and Rwandan communities, respectively, led to increased employment, educational attainment, and business activity. Our study expands upon this result by examining a broader set of questions that capture economic productivity both at the household and business level.}
The responses from open-ended survey questions provided additional depth and nuance to these findings. The availability of electronic information devices such as televisions and radios has expanded access to information for residents, while the convenience of charging phones has enhanced communication within families and the community at large. The ability to access social media platforms has also heightened residents' awareness of local, national, and international news. Furthermore, the acquisition of electric appliances like clothes irons has enabled residents to wear ironed clothes, contributing to an improved sense of self-presentation.
There was also an improvement in the academic performance of school-aged children affirmed by both households and schools. Children could now extend their study hours at home, and schools could provide more services to support the student body, potentially at later hours. Most organizations also had a notable increase in their hours of operation---now with a reliable source of electricity---thus driving more business growth in hiring more workers and expanding their lines of products and services.

Sixteen health-related questions were posed to households, focusing on the ease of access to the nearest clinic, use of kerosene lamps, water purification methods, and the overall quality of healthcare. The results highlighted a significant positive impact on the health and well-being of communities connected to mini-grids. Residents gained access to clean drinking water, primarily through community wells or pumps, reducing health risks associated with the use of unclean water.
Clinics reported shorter wait times, extended hours of operation, the capacity to treat more patients, and improved cold storage for vaccines and medicines. Residents directly confirmed an overall enhancement in healthcare services and living conditions.
The adoption of mini-grids led many households to discontinue the use of hazardous kerosene lamps, known for their risks of poisoning, fires, and explosions. Here, too, open-ended survey responses further enriched the findings. The consistent voltage supply from the mini-grid has enabled a shift away from petrol-powered generators, significantly reducing noise and pollution in residential areas. Residents, like clinics, also benefit from cold storage facilities, allowing access to cold drinking water and other chilled beverages, alongside a notable reduction in food wastage. This transition has contributed to an improved quality of life for the residents, underscoring the multifaceted health and lifestyle benefits brought about by the mini-grid installation.

The safety of residents was assessed based on how safe they felt and any potential reasons they might feel unsafe, including due to theft, unsafe travel, and lack of community lighting, with 10 questions allocated from the surveys. Households were more likely to have exterior lighting, thus contributing to a feeling of safety in and around the house. On the feeling of safety, there was no significant increase or decrease, thus indicating that most residents did not affirm feeling more or less safe. However, there was a notable and positive change in the reasons related to electrification. Indeed, there was a drop in respondents denoting community lighting and lack of safety when traveling as reasons for feeling unsafe. Even without an overall increase in the feeling of safety, those two reasons indicate that the mini-grid installation reduced the incidence of electrification-related concerns. On the other hand, there was a significant increase in the perceived threat of theft following the mini-grid installation. This observation suggests a need for further investigation to assert a direct correlation with the mini-grid installation. Future work should avoid this limitation by expanding the number of questions dedicated to assessing safety.

Survey questions measuring economic activity assessed changes in household incomes and spending, earnings for businesses, employment, and business ownership. This section contained 17 questions. The median household income in Nigeria decreased between 2021 and 2023, but this can likely be best explained by macroeconomic trends \cite{wb-nigeria-2,wb-nigeria-1}. On the other hand, Kenyan households observed an increase in income after mini-grid connection.
Among the paired respondents---those community members who were present both before the mini-grid installation and at least until the post-survey one year after connection---there was a significant rise in median monthly income. This group saw their median income increase from 4,000 KES to 18,000 KES, highlighting a substantial economic uplift due to the mini-grid installation. \textcolor{purple}{This is in contrast to the findings of \cite{lee2020experimental}, which did not find meaningful impacts on economic and non-economic outcomes after the expansion of electric grid infrastructure in rural Kenya, potentially due to the low energy utilization they observed at the household level.}

Across both countries, business ownership was higher, and previously established businesses had increased earnings. The mini-grid installation bolstered the entrepreneurial economy as electrification brought additional economic opportunities. Households could also save on the charging cost for phones and appliances, cooking energy, and water costs, freeing up money for other investments.