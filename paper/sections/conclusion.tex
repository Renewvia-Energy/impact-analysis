\section{Conclusion}
\label{sec:conclusion}
This paper offers an in-depth evaluation of the tangible socioeconomic impacts of solar mini-grids in rural communities across sub-Saharan Africa. It utilizes robust empirical methods to examine five Key Performance Indicators (KPIs): gender equality, productivity, health, safety, and economic activity. This constitutes the first comprehensive analysis of the social and economic effects of solar mini-grids in rural African settings.

Surveys were conducted in Nigeria and Kenya at selected sites before and one year after solar mini-grid installation, yielding a detailed insight into the mini-grids' impacts. Generating both quantitative and qualitative data, the surveys provided a holistic perspective on various aspects such as household chores, education access, time use, healthcare, and incomes. Descriptive statistics and comparative tests, including the paired \textit{t}-test and McNemar's test, were used to evaluate significant differences pre- and post-installation.

The study showed marked improvements in children's schooling, including better academic performance and less involvement in household chores, indicating that mini-grids enhanced educational opportunities and potentially raised literacy rates. Additional lighting hours and reduced time for water and cooking fuel collection from the mini-grid led to increased productivity. Additionally, replacing hazardous kerosene lamps with the cleaner, reliable power of the mini-grid improved both community health and productivity.

The mini-grid's on-demand home electricity significantly reduced household expenses, notably in services like phone charging. This installation led to diverse improvements, positively affecting many aspects of community life.

Despite the study's positive findings, certain limitations were encountered. The survey's structure led to an imbalance between discrete and continuous variables, limiting the pool available for normal distribution in regression analyses. This restriction affected the robustness of testing electricity consumption effects at the household level. Site-level analyses provided some compensation, but they offered only community-wide results, amalgamating individual household and business impacts. Furthermore, many survey questions yielded outputs unsuitable for parametric hypothesis testing due to their lack of meaningful order or magnitude.

This study's insights on solar mini-grids in Nigeria and Kenya pave the way for future research. Longitudinal studies tracking participants over years would deepen understanding of mini-grids' long-term effects. Comparing these communities with nearby unelectrified ones could create a natural quasi-experiment. Broadening the study to other regions would enhance understanding of mini-grids' diverse impacts. Assessing how evolving renewable technologies affect these impacts, and examining the influence of government policies, subsidies, and international aid, are also crucial. Addressing these aspects will enrich the knowledge base, aiding the effective use and optimization of solar mini-grids for greater social and economic advantages.

Solar mini-grids have significantly driven positive transformations in communities, leading to enhanced opportunities for women and girls, better healthcare, and economic growth. Their impact on various socioeconomic factors highlights their role in achieving the Sustainable Development Goals (SDGs) by 2030.